\chapter{Metodologia}

\section{Problema direto}

Seja $\mathbf{d}^{o}$ o vetor de dados observados, cujo $i$-ésimo elemento $d^{o}_{i}$, $i = 1, \dots, N$, é a anomalia de campo total observada (Equação \ref{eq:tfanomaly-general}) no ponto $(x_{i}, y_{i}, z_{i})$.
Considere que as anomalias de campo total produzidas por pequenas fontes magnéticas não-alvo podem distorcer localmente a anomalia causada por uma fonte alvo 3D.
Adicionalmente considere que o campo geomagnético principal é constante na área de estudo, com declinação $ D_ {0} $ e inclinação $ I_ {0} $.
Este trabalho segue a mesma abordagem apresentada por \cite{oliveirajr_etal2011} e \cite{oliveirajr_barbosa2013} para definir o modelo interpretativo que aproxima a geometria da fonte alvo.
Esse modelo é formado por um conjunto de $L$ prismas retos verticalmente justapostos tendo a mesma espessura $dz$ e o mesmo vetor de magnetização total com intensidade $m_{0}$, declinação $D$ e inclinação $I$ (Figura \ref{fig:schematic}).

A profundidade do topo do prisma mais raso é definida por $z_{0}$.
Cada prisma possui a seção horizontal definida por um polígono com $V$ vértices igualmente espaçados de $0^{\circ}$ a $360^{\circ}$.
As posições horizontais dos vértices que formam o $k$-ésimo prisma
são definidas por distâncias radiais (ou apenas raios) $r^{k}_{j}$, com respeito a uma origem $(x_{0}^{k}, y_{0}^{k})$ localizada dentro do prisma, $k = 1, \dots, L$, $j = 1, \dots, V$ (Figura \ref{fig:pk}).
A anomalia de campo total predita pelo modelo interpretativo no ponto $(x_{i}, y_{i}, z_{i})$, $i = 1, \dots, N$, é dada por:
\begin{equation}
d_{i} (\mathbf{p}) \equiv \sum\limits_{k=1}^{L} f_{i}^{k}(\mathbf{r}^{k}, x_{0}^{k}, y_{0}^{k}, dz, z_{1}^{k}, m_{0}, D, I, D_{0}, I_{0}) \: ,
\label{eq:predicted-data-i}
\end{equation}
em que $\mathbf{r}^{k}$ é um vetor de dimensão $V \times 1$ que contém os raios $r^{k}_{j}$ dos vértices pertencentes ao $k$-ésimo prisma, que possui origem no ponto $(x_{0}^{k}, y_{0}^{k})$ e profundidade do topo em $z_{1}^{k} = z_{0} + (k-1)dz$.
Na Equação \ref{eq:predicted-data-i}, $\mathbf{p}$ é um vetor de parâmetros de dimensão $M \times 1$, $ M=L(V+2) +1 $, que define a geometria do modelo interpretativo:
%\begin{equation}
%\mathbf{p} = \left[ \begin{array}{@{}*{12}{c}@{}}
%{\mathbf{r}^{1}}^{\mathsf{T}} & x_{0}^{1} & y_{0}^{1} & \dots & {\mathbf{r}^{L}}^{\mathsf{T}} & x_{0}^{L} & y_{0}^{L} & dz \\
%\end{array} \right]^{\mathsf{T}} \: .
%\label{eq:p-vector}
%\end{equation}
\begin{equation}
\mathbf{p} = \begin{bmatrix} 
{\mathbf{r}^{1}}^{\top} & x_{0}^{1} & y_{0}^{1} & \dots & 
{\mathbf{r}^{L}}^{\top} & x_{0}^{L} & y_{0}^{L} & dz
\end{bmatrix}^{\top} \: .
\label{eq:p-vector}
\end{equation}
A anomalia de campo total $d_{i} (\mathbf{p})$ 
(Equação \ref{eq:predicted-data-i}) é computada por meio das fórmulas de \cite{plouff1976} implementadas no pacote de Python 
Fatiando a Terra \citep{uieda-etal2013}.

%FIGURA
\begin{figure}[!htb]
	\centering
	\includegraphics[scale=0.35]{schematic.png}
	\caption{Representação esquemática do modelo interpretativo. (a) Anomalia de campo total produzida por uma fonte magnética 3D localizada em subsuperfície (volume cinza escuro em b). (b) Modelo interpretativo formado por $ L $ prismas retos $ P^k$, $k = 1, \dots , L $ (prismas cinza claro), verticalmente justapostos e com seção horizontal descrita por um polígono. A profundidade do topo $z_0$ do modelo interpretativo coincide com a da fonte magnética (volume cinza escuro).}
	\label{fig:schematic}
\end{figure}

%FIGURA
\begin{figure}[!htb]
	\centering
	\includegraphics[scale=0.45]{pk.png}
	\caption{Representação esquemática do $k$-ésimo prisma $P^k$, $k=1,\dots, L$, que compõe o modelo interpretativo (Figura 1b). Este prisma tem espessura dz, profundidade do topo $z_1^k$ e seção horizontal descrita por um polígono com $V$ vértices igualmente espaçados entre $0^{\circ}$ e $360^{\circ}$.
}
	\label{fig:pk}
\end{figure}
\pagebreak

\section{Problema inverso}

Este trabalho propõe um método robusto de inversão magnética para estimar a posição e a forma de uma fonte magnética alvo 3D, que pode ou não estar na presença de fontes não-alvo.
A fonte alvo é aquela que gera a anomalia de campo total de maior extensão horizontal e que é geralmente a mais intensa.
Já as fontes não-alvo não aquelas que produzem um sinal interferente menos intenso que ocupa uma área menor em relação à fonte principal.
O método foi formulado como um problema de otimização não-linear vinculado para estimar um vetor de parâmetros $\mathbf{p}$ (Equação \ref{eq:p-vector}) minimizando a função objetivo
\begin{equation}
\Gamma (\mathbf{p}) = \phi (\mathbf{p}) + \sum\limits^{5}_{\ell =1} \alpha_{\ell} \, \varphi_{\ell}(\mathbf{p}) \: ,
\label{eq:gamma}
\end{equation}
sujeito aos vínculos de desigualdade
\begin{equation}
p_{l}^{min} < p_{l} < p_{l}^{max}, \quad l = 1, \dots, M \: ,
\label{eq:inequality-constraints}
\end{equation}
em que $p_{l}^{min}$ e $p_{l}^{max}$ definem, respectivamente, os limites inferior e superior para o $l$-ésimo elemento $p_{l}$ do vetor de parâmetros $\mathbf{p}$ (Fig. \ref{fig:barreira}),
$\varphi_{\ell}(\mathbf{p})$ são as funções que representam os vínculos que impõem informação a priori sobre a forma da estimativa do corpo 3D, e $\phi (\mathbf{p})$ 
é a função desajuste dos dados -- ou \textit{data-misfit} em inglês.

%FIGURA
\begin{figure}[!htb]
	\centering
	\includegraphics[width=0.75\linewidth]{Constraint_barrier.png}
	\caption{Representação esquemática dos vínculos de desigualdade. A figura exibe os prisma $P^k$ e os intervalos de máximo e mínimo de $r^k_j$, $x_0$, $y_0$ e $ dz $.}
	\label{fig:barreira}
\end{figure}


Podemos definir $\phi (\mathbf{p})$ através de duas abordagens diferentes com o propósito de comparar os resultados. Na primeira abordagem, $\phi (\mathbf{p})$ como
\begin{equation}\label{eq:L2_misfit}
\phi (\mathbf{p}) = \frac{1}{N} 
\left\| \mathbf{d}^{o} - \mathbf{d}(\mathbf{p}) \right\|_{2}^{2} \quad ,
\end{equation}
que é a norma-2 quadrática \citep[por exemplo,][p. 331]{aster_etal2019} dos resíduos entre o vetor de dados observados $\mathbf{d}^{o}$, cujo $i$-ésimo elemento $d_{i}^{o}$ representa a anomalia de campo total observada no ponto $(x_{i}, y_{i}, z_{i})$, e o vetor de dados preditos $\mathbf{d}(\mathbf{p})$, cujo $i$-ésimo elemento $d_{i} (\mathbf{p})$ é definido pela Equação \ref{eq:predicted-data-i}.
Alternativamente, podemos definir a função \textit{data-misfit} como
\begin{equation}\label{eq:L1_misfit}
\phi (\mathbf{p}) = \frac{1}{N} 
\left\| \mathbf{d}^{o} - \mathbf{d}(\mathbf{p}) \right\|_{1} \quad ,
\end{equation}
que representa a norma-1 \citep[por exemplo,][p. 331]{aster_etal2019}
dos resíduos entre os vetores de dados observados $\mathbf{d}^{o}$ e preditos $\mathbf{d}(\mathbf{p})$.
É de amplo conhecimento que o vetor de parâmetros que minimiza a norma-2 quadrática (Equação \ref{eq:L2_misfit}) pode sofrer distorções causadas pela presença de \textit{outliers} e também pelo efeito causado por fontes não-alvo \cite[por exemplo,][]{claerbout_muir1973, 
	silva_hohmann1983, scales_gersztenkorn1988, silva_cutrim1989, farquharson_oldenburg1998, 
	uieda_barbosa2012, oliveirajr_etal2015, aster_etal2019}.
Através da estimativa do vetor de parâmetros obtida pela minimização da norma-1 (Equação \ref{eq:L1_misfit}), espera-se que a posição e a forma estimadas do corpo 3D durante a inversão ajustem a anomalia de campo total produzida pela fonte alvo e ignorem a causada pelas fontes não-alvo.

Na Equação \ref{eq:gamma}, $\alpha_{\ell}$, $\ell = 1, \dots, 5$, são escalares positivos que definem o peso relativo das funções dos vínculos $\varphi_{\ell}(\mathbf{p})$.
Essas funções são definidas seguindo a mesma abordagem utilizada por \citet{oliveirajr_etal2011} e \citet{oliveirajr_barbosa2013}.

\section{Vínculos}\label{sec:constraints}

As funções dos vínculos $\varphi_{\ell}(\mathbf{p})$ (Equação \ref{eq:gamma}), $\ell = 1, \dots, 5$, utilizadas aqui para obter soluções estáveis e introduzir informação a priori sobre o corpo estimado, foram organizadas em dois grupos para um melhor entendimento.

\subsection{Vínculos de suavidade}

Este grupo é formado pelas variações da regularização de Tikhonov de primeira ordem \cite[][ p. 103]{aster_etal2019} que impõe suavidade sobre os raios $r_{j}^{k}$ e sobre as coordenadas Cartesianas $x_{0}^{k}$ e $y_{0}^{k}$ da origem $O^{k}$, $j = 1, \dots, V$, $k = 1, \dots, L$, que define a seção horizontal de cada prisma (Fig.\ref{fig:schematic}b).
Elas foram propostas por \cite{oliveirajr_etal2011} e \cite{oliveirajr_barbosa2013} e possuem um papel muito importante em introduzir informação a priori sobre a forma da fonte alvo. 

O primeiro vínculo deste grupo é a \textit{suavidade sobre os raios adjacentes que definem a seção horizontal de cada prisma}. Esse vínculo impõe que os raios adjacentes $r_{j}^{k}$ e $r_{j+1}^{k}$ dentro do mesmo prisma devam ter comprimento semelhantes. Isso força uma forma aproximadamente cilíndrica de cada prisma estimado, evita descontinuidades abruptas entre as estimativas das distâncias radiais dentro de um mesmo prisma. Sua representação esquemática é mostrada na Figura \ref{fig:phi1}.

Matematicamente, o vínculo é dado por \begin{equation}\label{eq:phi1}
\begin{split}
\varphi_{1}(\mathbf{p}) &= \sum\limits^{L}_{k=1}\left[\left(r^{k}_{V}-r^{k}_{1}\right)^2 + \sum\limits^{V-1}_{j=1}\left(r^{k}_{j}-r^{k}_{j+1}\right)^2\right]\\
&= \mathbf{p}^{\mathsf{T}} \mathbf{R}^{\mathsf{T}}_{1}\mathbf{R}_{1} \mathbf{p} \quad ,
\end{split}
\end{equation}
em que
\begin{equation}
\mathbf{R}_{1} = 
\begin{bmatrix}
\mathbf{S}_{1} &
\mathbf{0}_{LV \times 1} \\
\end{bmatrix}_{LV \times M} \quad ,
\label{eq:R1-matrix}
\end{equation}
\begin{equation}
\mathbf{S}_{1} = 
\mathbf{I}_{L} \otimes 
\begin{bmatrix}
\left( \mathbf{I}_{V} - \mathbf{D}_{V}^\mathsf{T} \right) & \mathbf{0}_{V \times 2} \\
\end{bmatrix}_{V \times (V+2)} \quad ,
\label{eq:S1-matrix}
\end{equation}
$ \mathbf{0}_{LV \times 1} $ é um vetor com $ LV \times 1 $ elementos nulos, $\mathbf{I}_{L}$ é a matriz identidade de ordem $L$, ``$\otimes$" indica o produto de Kronecker \cite{}\cite[][ p. 243]{horn_johnson1991}, $\mathbf{0}_{V \times 2}$ é uma matriz de ordem $V \times 2$ com elementos nulos, 
$\mathbf{I}_{V}$ é a matriz identidade de ordem $V$ e $\mathbf{D}_{V}^\mathsf{T}$ é a matriz de permutação superior de ordem $V$ \cite[][ p. 20]{golub-vanloan2013}. O vetor gradiente e a matriz Hessiana da função $\varphi_{1}(\mathbf{p})$ (Equação \ref{eq:phi1}) são dados por:
\begin{equation}\label{eq:phi1_gh}
\begin{split}
\boldsymbol{\nabla}\varphi_{1}(\mathbf{p}) &= 2 \mathbf{R}^\mathsf{T}_{1}\mathbf{R}_{1}\mathbf{p} \quad , \\
\mathbf{H}_{1}(\mathbf{p}) &= 2\mathbf{R}^\mathsf{T}_{1}\mathbf{R}_{1} \quad .
\end{split}
\end{equation}

%FIGURA
\begin{figure}[!htb]
	\centering
	\includegraphics[scale=0.45]{Constraint_phi1.png}
	\caption{Representação esquemática do vínculo de suavidade sobre distâncias adjacentes dentro de um mesmo prisma $\varphi_{1}$. A figura exibe o k-ésimo prisma $P^k$ e as distâncias radiais adjacentes $r_j^k$ e $r_{j+1}^k$ relacionadas ao vínculo.}
	\label{fig:phi1}
\end{figure}

O segundo vínculo do grupo é a \textit{suavidade sobre os raios adjacentes de prismas adjacentes}, o qual impõe que os raios adjacentes $r_{j}^{k}$ e $r_{j}^{k+1}$ entre prismas verticalmente adjacentes tenham comprimentos semelhantes. Esse vínculo força que a forma de prismas verticalmente adjacentes seja similar. Uma representação esquemática do vínculo é apresentada na Figura \ref{fig:phi2}.

De forma matemática o vínculo é dado por
\begin{equation}\label{eq:phi2}
\begin{split}
\varphi_{2}(\mathbf{p}) &= \sum\limits^{L-1}_{k=1}\left[\sum\limits^{V}_{j=1}\left(r^{k+1}_{j}-r^{k}_{j}\right)^2\right] \\
&= \mathbf{p}^{\mathsf{T}} \mathbf{R}^{\mathsf{T}}_{2}\mathbf{R}_{2}\mathbf{p}
\end{split} \quad ,
\end{equation}
em que
\begin{equation}
\mathbf{R}_{2} = 
\begin{bmatrix}
\mathbf{S}_{2} & \mathbf{0}_{(L-1)V \times 1} \\
\end{bmatrix}_{(L-1)V \times M} \quad ,
\label{eq:R2-matrix}
\end{equation}
\begin{equation}
\mathbf{S}_{2} =
\left( 
\begin{bmatrix} \mathbf{I}_{L-1} & \mathbf{0}_{(L-1) \times 1} \end{bmatrix} -
\begin{bmatrix} \mathbf{0}_{(L-1) \times 1} & \mathbf{I}_{L-1} \end{bmatrix} 
\right) \otimes 
\begin{bmatrix} \mathbf{I}_{V} & \mathbf{0}_{V \times 2} \end{bmatrix} \quad ,
\label{eq:S2-matrix}
\end{equation}
$\mathbf{0}_{(L-1)V \times 1}$ é um vetor de ordem $(L-1)V \times 1$ com elementos nulos,
$\mathbf{0}_{(L-1) \times 1}$ é um vetor de ordem $(L-1) \times 1$ com elementos nulos e 
$\mathbf{I}_{L-1}$ é a matriz identidade de ordem $L-1$. O vetor gradiente e a matriz Hessiana da função $\varphi_{2}(\mathbf{p})$ (Equação \ref{eq:phi2}) são dados por:
\begin{equation}\label{eq:phi2_gh}
\begin{split}
\boldsymbol{\nabla}\varphi_{2}(\mathbf{p}) &= 2\mathbf{R}^\mathsf{T}_{2}\mathbf{R}_{2}\mathbf{p} \quad ,\\
\mathbf{H}_{2}(\mathbf{p}) &= 2\mathbf{R}^\mathsf{T}_{2}\mathbf{R}_{2} \quad .
\end{split}
\end{equation}

%FIGURA
\begin{figure}[!htb]
	\centering
	\includegraphics[scale=0.45]{Constraint_phi2.png}
	\caption{Representação esquemática do vínculo de suavidade sobre distâncias adjacentes pertencentes a prismas adjacentes $\varphi_{2}$. A figura exibe o k-ésimo prisma $P^k$ e seu adjacente $P^{k+1}$, assim como as distâncias radiais adjacentes $r_j^k$ e $r_j^{k+1}$ relacionadas ao vínculo.}
	\label{fig:phi2}
\end{figure}

O último vínculo deste grupo é a \textit{suavidade sobre a posição horizontal das origens arbitrárias de prismas verticalmente adjacentes}. Esse vínculo impõe que as coordenadas Cartesianas horizontais estimadas $(x_{0}^{k}, y_{0}^{k})$ e $(x_{0}^{k+1}, y_{0}^{k+1})$ das origens $O^{k}$ e $O^{k+1}$ 
de prismas verticalmente adjacentes devem ser próximas entre si. Isso controla o mergulho do corpo estimado através da regularização do deslocamento horizontal de prismas verticalmente adjacentes (Figura \ref{fig:phi3}).

Algebricamente o vínculo é dado por
\begin{equation}\label{eq:phi3}
\begin{split}
\varphi_{3}(\mathbf{p}) &= \sum\limits^{L-1}_{k=1}\left[\left(x_{0}^{k+1} - x_{0}^{k}\right)^2 + \left(y_{0}^{k+1} - y_{0}^{k}\right)^2 \right] \\
&= \mathbf{p}^{\mathsf{T}} \mathbf{R}^{\mathsf{T}}_{3}\mathbf{R}_{3}\mathbf{p}
\end{split} \quad ,
\end{equation}
em que 
\begin{equation}
\mathbf{R}_{3} = 
\begin{bmatrix}
\mathbf{S}_{3} & \mathbf{0}_{(L-1)2 \times 1} \\
\end{bmatrix}_{(L-1)2 \times M} \quad ,
\label{eq:R3-matrix}
\end{equation}
\begin{equation}
\mathbf{S}_{3} =
\left( 
\begin{bmatrix} \mathbf{I}_{L-1} & \mathbf{0}_{(L-1) \times 1} \end{bmatrix} -
\begin{bmatrix} \mathbf{0}_{(L-1) \times 1} & \mathbf{I}_{L-1} \end{bmatrix} 
\right) \otimes 
\begin{bmatrix} \mathbf{0}_{2 \times V} & \mathbf{I}_{2} \end{bmatrix} \quad ,
\label{eq:S3-matrix}
\end{equation}
$\mathbf{0}_{(L-1)2 \times 1}$ é um vetor de ordem $(L-1)2 \times 1$ com elementos nulos,
$\mathbf{0}_{2 \times V}$ é uma matrix de ordem $2 \times V$ com elementos nulos e 
$\mathbf{I}_{2}$ é uma matriz identidade de ordem $2$. O vetor gradiente e a matriz Hessiana da função $\varphi_{3}(\mathbf{p})$ (Equação \ref{eq:phi3}) são dados por:
\begin{equation}\label{eq:phi3_gh}
\begin{split}
\boldsymbol{\nabla}\varphi_{3}(\mathbf{p}) &= 2\mathbf{R}^\mathsf{T}_{3}\mathbf{R}_{3}\mathbf{p} \quad ,\\
\mathbf{H}_{3}(\mathbf{p}) &= 2\mathbf{R}^\mathsf{T}_{3}\mathbf{R}_{3} \quad .
\end{split}
\end{equation}

%FIGURA
\begin{figure}[!htb]
	\centering
	\includegraphics[scale=0.45]{Constraint_phi5.png}
	\caption{Representação esquemática do vínculo de suavidade nas coordenadas das origens pertencentes a prismas adjacentes $\varphi_{3}$. A figura exibe os prismas $P^k$ e $P^{k+1}$ e suas respectivas coordenadas Cartesianas $(x_0^k,y_0^k)$, referidas à origem $O^k$, e $(x_0^{k+1},y_0^{k+1})$, referidas à origem $O^{k+1}$.}
	\label{fig:phi3}
\end{figure}

\subsection{Vínculos de norma Euclidiana mínima}

Dois vínculos utilizam a regularização Tikhonov de ordem zero com o propósito de estabilizar de maneira puramente matemática o problema inverso sem necessariamente introduzir informação a priori com significado físico sobre a fonte. 

A \textit{norma Euclidiana mínima dos raios} impõe que todos os raios estimados dentro de um prisma devem ser próximos de zero (Figura \ref{fig:phi4}).

Esse vínculo foi proposto por \cite{oliveirajr_etal2011} e \cite{oliveirajr_barbosa2013} e pode ser reescrito como
\begin{equation}\label{eq:phi4}
\begin{split}
\varphi_{4}(\mathbf{p}) &= \sum\limits^{L}_{k=1}\sum\limits^{V}_{j=1}\left(r_{j}^{k}\right)^2 \\
&= \mathbf{p}^{\mathsf{T}} \mathbf{R}_{4}^{\mathsf{T}} \mathbf{R}_{4} \mathbf{p}
\end{split} \quad ,
\end{equation}
em que
\begin{equation}
\mathbf{R}_{4} = 
\begin{bmatrix}
\mathbf{S}_{4} & \mathbf{0}_{(M-1) \times 1} \\
\mathbf{0}_{1 \times (M-1)} & 0 \\
\end{bmatrix}_{M\times M} \quad ,
\label{eq:R4-matrix}
\end{equation}
e
\begin{equation}
\mathbf{S}_{4} = 
\mathbf{I}_L \otimes
\begin{bmatrix}
\mathbf{I}_{V} & \mathbf{0}_{V \times 2} \\
\mathbf{0}_{2 \times V} & \mathbf{0}_{2\times 2} \\
\end{bmatrix}_{ (V+2)\times (V+2)} \quad .
\label{eq:S4-matrix}
\end{equation}
O vetor gradiente e a matriz Hessiana da função $\varphi_{4}(\mathbf{p})$ (Equação \ref{eq:phi4}) são:
\begin{equation}\label{eq:phi4_gh}
\begin{split}
\boldsymbol{\nabla}\varphi_{4}(\mathbf{p}) &= 2\mathbf{R}^\mathsf{T}_{4}\mathbf{R}_{4}\mathbf{p} \quad ,\\
\mathbf{H}_{4}(\mathbf{p}) &= 2\mathbf{R}^\mathsf{T}_{4}\mathbf{R}_{4} \quad .
\end{split}
\end{equation}

%FIGURA
\begin{figure}[!htb]
	\centering
	\includegraphics[scale=0.45]{Constraint_phi6.png}
	\caption{Representação esquemática do vínculo de Tikhonov de ordem zero nas distâncias radiais de um prisma $\varphi_{4}$. A figura exibe os prisma $P^k$ e suas respectivas distâncias radiais $r_j^k$ referidas à origem $O^k$. O vínculo atua sobre as distâncias radiais do prismas, levando-as próximas a zero.}
	\label{fig:phi4}
\end{figure}

Finalmente, o último vínculo é a \textit{norma Euclidiana mínima da espessura}, que impõe que a espessura comum $ dz $ de todos os prismas seja próxima de zero. Esse vínculo força que a profundidade da base do modelo seja o mais rasa possível (Figura \ref{fig:phi5})

Esse vínculo pode ser escrito matematicamente como
\begin{equation}\label{eq:phi5}
\begin{split}
\varphi_{5}(\mathbf{p}) &= dz^2 \\
&= \mathbf{p}^{\mathsf{T}} \mathbf{R}_{5}^{\mathsf{T}} \mathbf{R}_{5} \mathbf{p}
\end{split} \quad ,
\end{equation}
em que
\begin{equation}
\mathbf{R}_{5} =
\begin{bmatrix}
\mathbf{0}_{(M-1) \times (M-1)} & \mathbf{0}_{(M-1) \times 1} \\
\mathbf{0}_{1 \times (M-1)} & 1 \\
\end{bmatrix}_{ M \times M } \quad .
\end{equation}
O vetor gradiente e a matriz Hessiana da função $\varphi_{5}(\mathbf{p})$ (Equação \ref{eq:phi5}) são:
\begin{equation}\label{eq:phi5_gh}
\begin{split}
\boldsymbol{\nabla}\varphi_{5}(\mathbf{p}) &= 2\mathbf{R}^\mathsf{T}_{5}\mathbf{R}_{5}\mathbf{p} \quad ,\\
\mathbf{H}_{5}(\mathbf{p}) &= 2\mathbf{R}^\mathsf{T}_{5}\mathbf{R}_{5} \quad .
\end{split}
\end{equation}

%FIGURA
\begin{figure}[!htb]
	\centering
	\includegraphics[scale=0.35]{Constraint_phi7.png}
	\caption{Representação esquemática do vínculo de Tikhonov de ordem zero $\varphi_{5}$ na espessura $dz$ dos prismas. A figura exibe os prisma $P^k$ e sua espessura. O vínculo atua sobre a espessura de os todos prismas levando-a próxima a zero, uma vez que $dz$ é igual para todos os prismas.}
	\label{fig:phi5}
\end{figure}

\section{Algoritmo de inversão}

Dada uma profundidade do topo do prisma mais raso $z_{0}$, a intensidade de magnetização total $m_{0}$ de todos os prismas, uma aproximação inicial $\hat{\mathbf{p}}_{(0)}$ para o vetor de parâmetros $\mathbf{p}$ (Equação \ref{eq:p-vector}), e os limites 
$p_{l}^{min}$ e $p_{l}^{max}$ (Equação \ref{eq:inequality-constraints}), o método de Levenberg-Marquardt \cite[por exemplo, ][ p. 624]{seber_wild2003} é utilizado para estimar o vetor de parâmetros $\hat{\mathbf{p}}^{\ast}$ que minimiza a função objetivo $\Gamma (\mathbf{p})$ (Equação \ref{eq:gamma}), sujeita aos vínculos de desigualdade definidos pela Equação \ref{eq:inequality-constraints}.
Para incorporar esses vínculos de desigualdade, foi empregada a mesma abordagem apresentada por \cite{barbosa_etal1999}, \cite{oliveirajr_etal2011} e \cite{oliveirajr_barbosa2013}.
Abaixo, segue o algoritmo de inversão aqui proposto:

\begin{itemize}
	\item[\textbf{entrada}] $\mathbf{d}^{o}$, $D_{0}$, $I_{0}$, $z_{0}$, 
	$m_{0}$, $D$, $I$, $p_{l}^{min}$ e $p_{l}^{max}$ (Equação 
	\ref{eq:inequality-constraints}), $k = 0$, $\hat{\mathbf{p}}_{(k)}$, e
	$\mathbf{W}_{(k)} = \mathbf{I}$, em que $\mathbf{I}$ é a matriz identidade de ordem $M$.
	\item[\textbf{(1)}] Computa a matriz $N \times M$ $\mathbf{G}(\hat{\mathbf{p}}_{(k)})$, cujo elemento $ij$ é a derivada do dado $d_{i}(\hat{\mathbf{p}}_{(k)})$ 
	(Equação \ref{eq:predicted-data-i}) com respeito ao $j$-ésimo elemento $p_{j}$ do vetor de parâme\-tros $\mathbf{p}$ (Equação \ref{eq:p-vector}):
	$$
	g_{ij} =  \dfrac{\partial d_{i}(\hat{\mathbf{p}}_{(k)})}{\partial{p_j}} \: .
	$$
	\item[\textbf{(2)}] Computa o vetor gradiente 
	$$
	\boldsymbol{\nabla} \phi(\hat{\mathbf{p}}_{(k)}) = 
	- \frac{2}{N} \mathbf{G}(\hat{\mathbf{p}}_{(k)})^{\top} 
	\mathbf{W}_{(k)} 
	\left[ \mathbf{d}^{o} - \mathbf{d}(\hat{\mathbf{p}}_{(k)}) \right]
	$$
	e a matriz Hessiana
	$$
	\mathbf{H}_{\phi}(\hat{\mathbf{p}}_{(k)}) = \frac{2}{N} 
	\mathbf{G}(\hat{\mathbf{p}}_{(k)})^{\top} \mathbf{W}_{(k)} 
	\mathbf{G}(\hat{\mathbf{p}}_{(k)})
	$$
	da função \textit{data-misfit} $\phi (\mathbf{p})$ (Equação \ref{eq:L2_misfit}),
	quando $\mathbf{W}_{(k)} = \mathbf{I}$, ou $\phi (\mathbf{p})$ 
	(Equação \ref{eq:L1_misfit}), quando $\mathbf{W}_{(k)} \ne \mathbf{I}$.
	Na próxima seção, será discutido como usar a matriz Hessiana 
	$\mathbf{H}_{\phi}(\hat{\mathbf{p}}_{(0)})$ (computada na iteração $k = 0$) 
	para definir os pesos $\alpha_{\ell}$ (Equação \ref{eq:gamma}) das funções de vínculos $\varphi_{\ell}(\mathbf{p})$ 
	(Equações \ref{eq:phi1}, \ref{eq:phi2}, \ref{eq:phi3}, \ref{eq:phi4} e \ref{eq:phi5}).
	\item[\textbf{(3)}] Computa o vetor gradiente 
	$$
	\boldsymbol{\nabla}\Gamma(\hat{\mathbf{p}}_{(k)}) = 
	\boldsymbol{\nabla}\phi (\hat{\mathbf{p}}_{(k)}) + 
	\sum\limits^{5}_{\ell =1} \alpha_{\ell} \, \boldsymbol{\nabla}\varphi_{\ell}(\hat{\mathbf{p}}_{(k)}) 
	$$ 
	e a matriz Hessiana
	$$
	\mathbf{H}(\hat{\mathbf{p}}_{(k)}) = 
	\mathbf{H}_\phi (\hat{\mathbf{p}}_{(k)}) + \sum\limits^{5}_{\ell =1} \alpha_{\ell} 
	\, \mathbf{H}_{\ell}
	$$ 
	da função objetivo $\Gamma (\mathbf{p})$ (Equação \ref{eq:gamma}),
	em que $\boldsymbol{\nabla}\varphi_{\ell}(\hat{\mathbf{p}}_{(k)})$ e
	$\mathbf{H}_{\ell}$ são, respectivamente, o vetor gradiente e a matriz Hessiana (Equações \ref{eq:phi1_gh}, \ref{eq:phi2_gh}, \ref{eq:phi3_gh}, \ref{eq:phi4_gh} e \ref{eq:phi5_gh}) das funções dos vínculos $\varphi_{\ell}(\mathbf{p})$ (Equações \ref{eq:phi1}, \ref{eq:phi2}, \ref{eq:phi3}, \ref{eq:phi4} e \ref{eq:phi5}).	
	\item[\textbf{(4)}] Computa o $l$-ésimo elemento $\hat{p}^{\dagger}_{l} \in (-\infty, +\infty)$ de um vetor não vinculado $\hat{\mathbf{p}}^{\dagger}_{(k)}$ como:
	$$
	\hat{p}^{\dagger}_{l} = -\ln\left(\frac{p_{l}^{max} - \hat{p}_{l}}{\hat{p}_{l} - p_{l}^{min}}\right) \: ,
	$$
	em que $\hat{p}_{l}$ é o $l$-ésimo elemento de $\hat{\mathbf{p}}_{(k)}$.
	\item[\textbf{(5)}] Computa uma matriz diagonal $\mathbf{T}(\hat{\mathbf{p}}_{(k)})$ 
	com o elemento $t_{ll}$ dado por
	$$
	t_{ll}(\hat{p}_{l}) = \frac{(p_{l}^{max} - \hat{p}_{l})(\hat{p}_{l} - p_{l}^{min})}{p_{l}^{max} - p_{l}^{min}} \: ,
	$$
	em que $p_{l}$ é o $l$-ésimo elemento do vetor $\hat{\mathbf{p}}_{(k)}$. Para evitar qualquer elemento da diagonal de $\mathbf{T}(\hat{\mathbf{p}}_{(k)})$ assuma um valor nulo, foi adicionada uma constante pequena e positiva de $ 10^{-10} $ a cada termo do numerador, seguindo a estratégia apresentada por \cite{barbosa_etal1999}.
	\item[\textbf{(6)}] Computa uma matriz 
	$$
	\mathbf{H}^{\dagger}(\hat{\mathbf{p}}_{(k)}) = \mathbf{H}(\hat{\mathbf{p}}_{(k)})\mathbf{T}(\hat{\mathbf{p}}_{(k)}) \: .
	$$
	\item[\textbf{(7)}] Computa uma matriz diagonal $\mathbf{Q}_{(k)}$ com elemento $q_{ll}$ dado por 
	$$
	q_{ll} = \frac{1}{\sqrt{h^{\dagger}_{ll}}} \: ,
	$$
	em que $h^{\dagger}_{ll}$ é o elemento $ll$ da matriz $\mathbf{H}^{\dagger}(\hat{\mathbf{p}}_{(k)})$ .
	\item[\textbf{(8)}] Computa uma correção 
	$\boldsymbol{\Delta}\hat{\mathbf{p}}^{\dagger}_{(k)}$ para o vetor 
	$\hat{\mathbf{p}}^{\dagger}_{(k)}$ pela solução do sistema linear
	$$
	\mathbf{Q}_{(k)}^{-1} \left[ \mathbf{Q}_{(k)} 
	\mathbf{H}^{\dagger}(\hat{\mathbf{p}}_{(k)}) \mathbf{Q}_{(k)} + 
	\lambda_{(k)} \mathbf{I}_{M} \right] \mathbf{Q}_{(k)}^{-1}
	\boldsymbol{\Delta} \hat{\mathbf{p}}^{\dagger}_{(k)} = 
	- \boldsymbol{\nabla}\Gamma(\hat{\mathbf{p}}_{(k)}) \: ,
	$$
	%	$$
	%	\left[ \mathbf{H}^{\dagger}(\hat{\mathbf{p}}_{(k)}) + 
	%	\lambda_{(k)} \mathbf{D}_{(k)} \right] \boldsymbol{\Delta} \hat{\mathbf{p}}^{\dagger}_{(k)} = 
	%	- \boldsymbol{\nabla}\Gamma(\hat{\mathbf{p}}_{(k)}) \: ,
	%	$$
	em que $\lambda_{(k)}$ é o parâmetro de Marquardt, um escalar positivo ajustado à cada iteração
	%	and $\mathbf{D}_{(k)}$ is a diagonal matrix formed by the diagonal elements 
	%	of $\mathbf{H}^{\dagger}(\hat{\mathbf{p}}_{(k)})$
	\citep[por exemplo,][p. 624]{seber_wild2003}.
	\item[\textbf{(9)}] Computa um novo vetor 
	%	\item[\textbf{(8)}] Compute a new vector 
	$$
	\hat{\mathbf{p}}^{\dagger}_{(k+1)} = \hat{\mathbf{p}}^{\dagger}_{(k)} + \boldsymbol{\Delta}\hat{\mathbf{p}}^{\dagger}_{(k)} \: .
	$$
	\item[\textbf{(10)}] Computa o $l$-ésimo elemento $ \hat{p}_{l}\in (p^{min}_l, p^{max}_l) $ do novo vetor vinculado
	%	\item[\textbf{(9)}] Compute the $l$th element of the new vector 
	$\hat{\mathbf{p}}_{(k+1)}$ como:
	$$
	\hat{p}_{l} = p_{l}^{min} + \left(\frac{p_{l}^{max} - p_{l}^{min}}{ 1 + e^{-\hat{p}^{\dagger}_{l}} }\right) \: .
	$$
	\item[\textbf{(11)}] Se o seguinte critério de convergência for satisfeito,
	%	\item[\textbf{(10)}] If the following convergence criterion is satisfied,
	$$
	\Bigg|
	\frac{\Gamma(\hat{\mathbf{p}}_{(k+1)}) - \Gamma(\hat{\mathbf{p}}_{(k)})}
	{\Gamma(\hat{\mathbf{p}}_{(k)})} 
	\Bigg| \le \tau \: ,
	$$ 
	em que $\tau$ é um número positivo pequeno, que varia de $\approx 10^{-3}$ a 
	$10^{-4}$, que controla a convergência, o vetor de parâmetros $\hat{\mathbf{p}}_{(k+1)}$ é a solução. 
	Senão, atualiza o vetor de parâmetros 
	$$
	\hat{\mathbf{p}}_{(k)} \leftarrow \hat{\mathbf{p}}_{(k+1)} \: ,
	$$
	atualiza o elemento $ii$ da matriz $\mathbf{W}_{(k)}$ para quando função de desajuste $ \phi $ é definida como a norma-1 (Eq. \ref{eq:L1_misfit})
	$$
	w_{ii} = \frac{1}{\mid d_{i}^{o} -d_{i}(\hat{\mathbf{p}}_{(k)}) \mid + 
		\, \varepsilon} \: ,
	$$
	em que $\varepsilon$ possui um valor positivo pequeno ($\approx 10^{-10}$) usado
	para prevenir uma divisão por zero, atualiza o contador da iteração $k$
	$$
	k \leftarrow k + 1 \: ,
	$$
	e retorna à etapa (1).
\end{itemize}

Neste algoritmo, os elementos da matriz $\mathbf{G}(\hat{\mathbf{p}}_{(k)})$ 
(etapa 1) são computados pelo uso das diferenças finitas centradas.
É importante notar que na etapa 3 as matrizes Hessianas $\mathbf{H}_{\ell}$ (Equações \ref{eq:phi1_gh}, \ref{eq:phi2_gh}, \ref{eq:phi3_gh}, \ref{eq:phi4_gh} e \ref{eq:phi5_gh})
das funções dos vínculos $\varphi_{\ell}(\mathbf{p})$ 
(Equações \ref{eq:phi1}, \ref{eq:phi2}, \ref{eq:phi3}, \ref{eq:phi4} e \ref{eq:phi5}) 
não dependem do vetor de parâmetros. Por essa razão, eles são computados apenas uma vez antes da primeira iteração e armazenados para serem usados até a convergência ser alcançada (etapa 11).

Este algoritmo é executado para obter um corpo estimado para cada ponto 
$(m_{0}, z_{0})$ em uma malha de valores de profundidade do topo $z_{0}$ e intensidade de magnetização total $m_{0}$ definida pelo usuário. 
Todos os corpos estimados são obtidos através da utilização de uma aproximação inicial $\hat{\mathbf{p}}_{(0)}$ para o vetor de parâmetros
$\mathbf{p}$ (Equação \ref{eq:p-vector}), dos mesmos valores para os pesos
$\alpha_{\ell}$ (Equação \ref{eq:gamma}) e dos limites $p_{l}^{min}$ e 
$p_{l}^{max}$ (Eq. \ref{eq:inequality-constraints}) para os parâmetros estimados.
Os valores ótimos da profundidade do topo $z_{0}$ e intensidade de magnetização total
$m_{0}$ são escolhidos como aqueles associados aos corpos estimados que produzem os menores valores da função objetivo $\Gamma (\mathbf{p})$ (Equação \ref{eq:gamma}).

Note que, ao manter a matriz $\mathbf{W}_{(k)}$ (etapa 2 e 10) igual à identidade ao longo das iterações, o corpo estimado minimiza a norma-2 quadrática dos resíduos entro os dados observados e preditos (Equação \ref{eq:L2_misfit}). 
Nesse caso, o corpo estimado é a solução L2. 
A atualização iterativa dos elementos da matriz $\mathbf{W}_{(k)}$ com os valores absolutos dos resíduos de acordo com a etapa 10 é feita através do método IRLS \citep[][p. 46]{scales_gersztenkorn1988, aster_etal2019} para obter um corpo estimado que minimiza a norma-1 dos resíduos entre os dados observados e preditos 
(Equação \ref{eq:L1_misfit}). Nesse caso, o corpo estimado é a solução L1.

