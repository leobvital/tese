\chapter*{Agradecimentos}

À minha mãe Antonia, que é meu alicerce, pelo esforço, dedicação e amor em minha criação. Sem seu apoio, nada disso teria acontecido.

Ao meu irmão Guido que sempre esteve presente e participou de muitas conquistas da minha vida.

Ao meu pai Arnaldo pelo seu carinho e a disposição em me ajudar sempre que possível.

Ao meu orientador Vanderlei pela dedicação em conduzir esta pesquisa como um parceiro, assim como à minha coorientadora professora Valéria que sempre esteve a disposição para ajudar. Não seria possível resumir em palavras tudo o que fizeram por mim. Sem eles esse trabalho não estaria finalizado.

Ao Observatório Nacional, CAPES e FAPERJ que forneceram as condições físicas e financeiras para o desenvolvimento deste trabalho e apresentações em congressos.

Aos membros da banca, Professora Dra. Yara, Professor Dr. Walter, Professora Dra. Kátia e Professor Dr. Cosme, por aceitarem o convite e contribuírem muito ao resultado final desta tese.

Ao Dr. Clive Foss que me recebeu muito bem no CSIRO, Sydney, e se tornou um amigo. 

Aos amigos e colegas do ON: Rodrigo Bijani, Fillipe Cláudio, Flora, Victor Carreira, Leo Miquelutti, Shayane, Dona Emília, Dona Ana, Cosme, Diogo LOC, Mário Martins, Walace, Mário Bicho, Bita, Simony, Plícida, Carol, Felipe Melo, Carlos André, Leo Uieda, Saulo, Birinho, Marlon, Wellington, Matias, Diogo Souto, Amanda, Diego Takahashi, Tiago, Hissa, Guga Rossi, Ximena, Cíntia, Cesar, Fábio.

Ao André que se tornou um irmão para mim ao longo de todos esses anos, desde o início da graduação em física na UERJ. Que venham muitos anos mais.

À minha companheira Larissa, cujo mais simples afago me impulsiona em antigravidade na imensidão de um amor terno e sereno.
