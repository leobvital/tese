\chapter{Introdução}

A interpretação de anomalias de campo total sobre a superfície da Terra é um desafio na geofísica de exploração devido à falta de unicidade da inversão magnética 3D. É bem estabelecido na literatura que diversas distribuições de magnetização em subsuperfície podem produzir o mesmo dado magnético com a mesma precisão. A fim de superar esta ambiguidade, é necessário introduzir informação a priori ao problema para reduzir o número de possíveis soluções que são coerentes com a geologia local. Existem basicamente três grupos de métodos de inversão magnética 3D. A informação a priori disponível determina a abordagem mais adequada para cada caso.

O primeiro grupo de métodos aproxima a fonte por um corpo causador geometricamente simples que possui geometria definida por um número pequeno de parâmetros \cite[por exemplo, ][]{ballantyne-1980,bhattacharyya-1980,silva-1983}. Esses métodos estimam tanto a geometria quanto a propriedade física da fonte através da solução de um problema inverso não linear. Devido à parametrização muito restritiva, tais métodos frequentemente sofrem graves problemas com a ambiguidade.

O segundo grupo é formado por uma vasta maioria dos métodos. Eles aproximam a subsuperfície por uma malha de prismas retangulares justapostos com uma direção de magnetização constante. Alguns métodos presumem uma magnetização puramente induzida \cite[por exemplo, ][]{cribb-1976,li_3-d_1996,pilkington_3-d_1997} e as susceptibilidades magnéticas isotrópicas dos prismas é a quantidade estimada pela solução de um problema inverso linear. Abordagens diferentes aperfeiçoaram este método de inversão para obter imagens da subsuperfície. Por exemplo, \cite{portniaguine_focusing_1999} e \cite{portniaguine_3d_2002} introduziram o auxílio do gradiente mínimo para minimizar o efeito de fortes variações e descontinuidades nos parâmetros pela inversão da anomalia magnética e qualquer componente do campo anômalo total. \cite{barbosa_silva2006} apresentaram um método para inverter anomalias magnéticas interferentes através da combinação de características da modelagem direta (a interatividade) e a inversão tradicional (o ajuste automático dos dados). 
Outros estudos introduziram estratégias para restringir a falta de unicidade e delinear a fonte \cite[]{tontini,pilkington_3d_2009,shamsipour_3d_2011,cella_inversion_2012,abedi-2015}. Excepcionalmente, alguns desses métodos permitem uma direção de magnetização diferente da direção do campo geomagnético principal \cite[por exemplo, ][]{pignatelli-2006}. Nesse caso, os parâmetros a serem estimados são as intensidades de magnetização total dos prismas. Em todos esses métodos, portanto, as geometrias das fontes magnéticas são indiretamente recuperadas interpretando a distribuição estimada de intensidade de magnetização total. Teoricamente, esses métodos de inversão são capazes de recuperar a geometria de fontes complexas. No entanto, eles exigem uma infinidade de informações a priori para superar sua falta de unicidade e instabilidade devido ao grande número de parâmetros a serem estimados. Além disso, essas técnicas são caracterizadas por um alto custo computacional associado à solução de grandes sistemas lineares.

O terceiro grupo de métodos de inversão magnética 3-D pressupõe algum conhecimento sobre a distribuição das propriedades físicas e estima a geometria das fontes.
Eles geralmente são formulados como problemas inversos não lineares. \cite{wang_inversion_1990} aproximam a fonte por um poliedro e estima a posição de seus vértices no domínio de Fourier. 
\cite{wenbin-2017} desenvolveram um método de múltiplos níveis para estimar a geometria de um conjunto de corpos causadores com susceptibilidade magnética uniforme. 
\cite{hidalgo-2019} invertem a anomalia de campo total para estimar a geometria do relevo do embasamento de uma bacia sedimentar com uma intensidade de magnetização conhecida mas direção desconhecida. 
Esse método possui um pequeno número de parâmetros a serem estimados por inversão e possui muito menos ambiguidade em comparação ao segundo grupo.

É de amplo conhecimento que problemas inversos que usam medidas de desajuste dos dados com base na norma-2 quadrática tradicional dos resíduos (ou soma dos quadrados dos resíduos), ou seja, inversões por mínimos quadrados, podem ser drasticamente afetados pela presença de pontos espúrios ou \textit{outliers} e também pelo efeito causado por fontes não-alvo (ruído geológico) nos dados observados.
Do ponto de vista estatístico, as inversões por mínimos quadrados não são robustas \citep{huber1964, scales_gersztenkorn1988}.
Para neutralizar esse problema, medidas robustas de desajuste dos dados, como a norma-1 (ou soma dos resíduos absolutos), o estimador-$ M $ \citep{huber1964}, ou a ``norma $ L_ {p} $ perturbada" \citep{ekblom1973} há muito tempo são usadas em problemas inversos em geofísica \citep{farquharson_oldenburg1998}.
Existem diversos métodos robustos para interpretar, por exemplo, dados sísmicos \citep[por exemplo,][]{claerbout_muir1973, scales_gersztenkorn1988, crase_etal1990, amundsen1991, guitton_symes2003, ji2012, dasilva2020} e também dados magnetotelúricos \citep[por exemplo,][]{egbert_booker1986, chave_etal1987, sutarno_vozoff1991, larsen_etal1996, matsuno_etal2014}.
Neste trabalho, o foco será sobre os métodos para inversão de dados de campos potenciais na presença de fontes não-alvo que usam medidas robustas da função desajuste dos dados ou uma abordagem semelhante.

\citet{silva_hohmann1983} apresentam uma inversão magnética não linear 2D para estimar a geometria de corpos simples via algoritmo de busca aleatória. Eles mostraram que as soluções obtidas usando a norma-1 são melhores do que a norma-2 quando há pequenas fontes não-alvo rasas. 
\citet{silva_cutrim1989} propõem um estimador de máxima probabilidade para inverter dados gravimétricos e magnéticos com base na suposição de erros que seguem uma distribuição de Cauchy.
Eles mostraram que seu método é mais robusto do que aqueles baseados nas normas 1 e 2 quando há a presença de ruído geológico.
\citet{beltrao_etal1991} desenvolveram uma abordagem robusta de mínimos quadrados reponderados iterativamente ou - do inglês \textit{Iteratively Reweighted Least Squares} - \citep{scales_gersztenkorn1988} para estimar os coeficientes de um polinômio que representa o campo gravitacional regional. 
Eles assumem que o campo residual não muda seu sinal em toda a área de estudo. 
\citet{barbosa_silva2006} introduziram uma inversão magnética 2D robusta para lidar com fontes não-alvo. Esse método estima a espessura das fontes magnéticas sobre um conjunto de elementos geométricos definidos pelo usuário, como segmentos de reta e pontos colocados em uma profundidade apropriada para representar as fontes não-alvo. 
Vale ressaltar que a robustez desse método não se deve ao uso de uma medida robusta de desajuste dos dados, mas ao seu esquema de concentração da distribuição de propriedade física estimada ao redor dos elementos geométricos. 
\citet{silva_dias_etal2007} propõem uma inversão de gravidade 2D baseada em uma abordagem IRLS para estimar a geometria de uma interface complexa entre dois meios contendo
heterogeneidades de densidade. 
\citet{uieda_barbosa2012} apresentam um método robusto de inversão do gradiente de gravidade para estimar a distribuição de densidade 3D em uma malha de prismas retangulares.
Seu método desenvolve sistematicamente a solução em torno de prismas especificados pelo usuário com valores de contraste de densidade predefinidos. Eles usam a função desajuste dos dados da norma-1 para ignorar o efeito de fontes não-alvo.
\citet{oliveirajr_etal2015} propõem um método IRLS robusto para estimar a direção da magnetização total de fontes magnéticas simples na presença de ruído geológico..

Este trabalho apresenta uma inversão magnética não linear robusta para estimar a geometria de uma fonte alvo 3D na presença ou não de fontes não-alvo.
Este método é uma extensão dos trabalhados apresentados por \citet{oliveirajr_etal2011} e 
\citet{oliveirajr_barbosa2013}, para inverter dados de gravidade e gradiometria gravimétrica, e o de \citet{vital_etal2019}, para inverter dados de anomalia de campo total.
Aproximamos a fonte por um modelo interpretativo formado por prismas retos justapostos verticalmente com seções horizontais definidas por polígonos, todos eles com o mesmo número de vértices.
Por conveniência, todos os prismas têm a mesma espessura e intensidade de magnetização total.
Da mesma forma que \citet{vital_etal2019}, este método estima a posição horizontal e a forma das seções horizontais de todos os prismas, bem como uma espessura constante comum para todos os prismas que formam o modelo de interpretativo.
Também presume-se que a fonte alvo tenha uma direção de magnetização total conhecida.
O método apresentado por \citet{vital_etal2019} geralmente falha em estimar corretamente a geometria de uma fonte alvo na presença de fontes não-alvo.
Essa limitação é uma consequência da estimativa corpos que utiliza a função de desajuste dos dados definida como a norma-2 ao quadrado dos resíduos entre os dados de anomalia de campo total observados e aqueles produzidos pelo modelo de interpretativo (dados preditos).
Por conveniência, chamamos esses corpos estimados de soluções de L2.
Para contornar essa limitação, diferentes corpos foram estimados, minimizando uma função objetivo composta de cinco funções de regularização e a função de desajuste dos dados que representada pela norma-1 dos resíduos entre os dados observados e preditos.
Os corpos estimados, convenientemente chamados de soluções L1, são obtidos a partir de um conjunto de valores especificados pelo usuário de profundidade do topo e intensidade de magnetização total.
A melhor solução L1 é definida como aquela que produz o menor valor da função objetivo e representa a melhor aproximação para a fonte alvo.
Este método também produz soluções L2 para comparação.
Da mesma forma que \citet{oliveirajr_etal2011}, \citet{oliveirajr_barbosa2013}, e \citet{vital_etal2019}, nossas funções de restrição são definidas usando regularização Tikhonov de ordem zero e de primeira ordem \citep[por exemplo,][p. 96 e 104]{aster_etal2019} com o objetivo de obter soluções estáveis e também introduzir informação a priori sobre a fonte alvo.

Este método foi aplicado para inverter dados sintéticos produzidos por um corpo geológico simples com forma de funil que satisfaz a maioria dos vínculos, um corpo inclinado para recuperar um corpo com um mergulho acentuado e a adição de um campo regional e um corpo complexo que exibe uma forma variável com a profundidade e viola alguns dos vínculos impostos neste trabalho.
Os corpos simulados representam as fontes alvo 3D.
Os resultados obtidos sem qualquer fonte não-alvo mostram que as melhores soluções L2 e L1 são muito semelhantes entre si e são capazes de recuperar satisfatoriamente a forma, a profundidade até o topo e a intensidade de magnetização total da fonte alvo.
O método também foi aplicado para inverter os dados magnéticos produzidos pela fonte alvo na presença de uma fonte não-alvo pequena e uma grande.
Em ambos os casos, a melhor solução L1 é superior à melhor solução L2 não apenas na recuperação da geometria da fonte alvo, mas também na estimativa da profundidade do topo e valores de intensidade de magnetização total mais próximos dos verdadeiros.
Os resultados com fontes não-alvo mostram que, ao usar a norma 1, o método é capaz de ignorar corretamente a anomalia de campo total interferente.

Este método foi aplicado para interpretar anomalias de campo total produzidas por complexos alcalinos brasileiros. Resultados com o complexo alcalino-carbonatítico de Anitápolis, SC, mostram uma intrusão inclinada aparentemente controlada por uma falha orientada aproximadamente a N30O, com profundidade do topo em $\approx 240$ m, intensidade de magnetização total $\approx 14$ A/m e extensão vertical de $\approx 3$ km. Esse resultado está de acordo com as informações a priori disponíveis na área de estudo e contribui para o debate sobre o controle estrutural do complexo de Anitápolis \citep{riccomini_etal2005, gomes_etal2018}.
Os resultados obtidos pela inversão dos dados magnéticos sobre o complexo Diorama, na Província Alcalina de Goiás, sugerem a presença de fontes não-alvo rasas no topo um pequeno corpo com profundidade do topo em $\approx 250$ m, forte intensidade de magnetização total $\approx 18$ A/m e extensão vertical de $\approx 2.4$ km. O corpo estimado parece ser influenciado por uma tendência quase N50E falha, que está em linha com a direção dos diques que cruzam o embasamento pré-cambriano na área de estudo \citet{junqueirabrod_etal2002}.