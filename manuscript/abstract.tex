\begin{foreignabstract}

\noindent A geological scenario usually includes multiple bodies that produce interfering geophysical signals.
The signals yielded by non-target sources can be considered geological noise and must be suppressed from the ones yielded by the sources classified by the interpreter as a target.
A target source can be considered, for example, as the source of the strongest geophysical signal, regardless of its economic value.
We have presented a robust method for inverting magnetic data to estimate the 3-D shape of a targeted source in the presence of non-targeted sources without previously filtering out the interfering signals.
By assuming knowledge of the total magnetization direction of the targeted source, our method retrieves its total magnetization intensity, position, and shape.
The method approximates the target source by a set of vertically juxtaposed right prisms, all with the same total magnetization vector and the same thickness.
The horizontal section of each prism is defined by a polygon that has a fixed number of equally spaced vertices from $ 0 ^ {\ circ} $ to $ 360 ^ {\ circ} $.
The position of the vertices, the horizontal location of each prism, and the thickness of the prisms are the parameters to be estimated during the inversion.
This is achieved by means of a regularized non-linear inversion in which the data-misfit function is defined by the 1-norm data residuals.
The estimated body in this case is conveniently called the L1 solution.
Tests on synthetic data show a better performance of the L1 solution when compared to the L2 solution (obtained by using a data-misfit function defined by the squared 2-norm of the data residuals) in retrieving the 3D shape of the targeted source in the presence of non-targeted sources.
In the lack of interfering signals, the L1 and L2 solutions show similar behavior.
Furthermore, the application on data produced by a synthetic inclined source with and without the influence of a regional field showed that both L1 and L2 solutions succeeded in estimating the shape of the targeted source.
Applications to real magnetic data on the Brazilian alkaline complexes of Anitápolis, southern Brazil, and the Diorama, central Brazil, suggest that both are controlled by failures consistent with information available in the literature.
Unlike the L1 and L2 solutions obtained for the Diorama complex, which suggest the presence of relatively large non-targeted sources, those obtained for the Anitápolis complex indicate the presence of small non-targeted sources.
These results show that the method can be a very efficient alternative in the interpretation of magnetic data contaminated with geological noise.

\end{foreignabstract}

