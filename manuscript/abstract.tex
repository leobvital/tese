\begin{foreignabstract}

\noindent A geologic scenario generally includes multiple geologic bodies producing interfering geophysical signals.
The signals yielded by non-targeted sources can be considered a geologic noise and should be suppressed from signals yielded by the targeted sources that give rise to the strongest effect.
Here, a targeted source is the one that gives rise to the strongest geophysical signal regardless of whether it is a geological target of high economic value.
We have presented a robust method for inverting magnetic data to estimate the 3-D shape of a targeted source in the presence of non-targeted sources without filtering out the interfering signals.
By assuming the knowledge of the total-magnetization direction of the targeted source, our method retrieves the total-magnetization intensity, the position, and the shape of a 3-D source.
We approximate the targeted source by an 
ensemble of vertically juxtaposed right prisms, all of them with the same 
total-magnetization vector and depth extent. The horizontal cross-section 
of each prism is defined by a polygon having the same number of vertices 
equally spaced from $0^{\circ}$ to $360^{\circ}$. The position of these 
vertices, the horizontal location of each prism, and the depth extent of all prisms 
are the parameters to be estimated by solving a constrained nonlinear inverse problem 
of minimizing a goal function. 
This is achieved by means of a non-linear regularized inversion in which the data-misfit function is defined by the 1-norm of the data residuals (L1 solution).
Tests on synthetic magnetic data show the better performance of the L1 solution 
when compared with the L2 solution (obtained by using the 2-norm squared of the data residuals) in retrieving the  3-D shape of a targeted source in the presence of non-targeted sources.
In the lack of interfering data, the L1 and L2 solutions are similar.
Furthermore, the application on data produced by an inclined source with or without the influence of strong regional field showed that both L1 and L2 solutions succeeded in estimating the shape of the targeted source.
Applications to real magnetic data on Brazilian alkaline complexes suggest that both the Anitápolis complex, southern Brazil, and the Diorama complex, central Brazil, are formed by non-targeted and targeted sources.
The geometry of the solutions suggests that the complexes are controlled by trending-faults compatible with information available in the literature.
Unlike the L1 and L2 solutions obtained for the Diorama complex, which suggest the presence of relatively large non-targeted sources, those obtained for the Anitápolis complex indicate the presence of small non-targeted sources.

\end{foreignabstract}

