\chapter{Conclusões}

Este trabalho apresentou uma inversão magnética regularizada não linear com um ajuste robusto a fim de estimar a forma 3D de uma fonte definida pelo intérprete como alvo na presença ou não de fontes não-alvo.
Aqui, uma fonte alvo é definida como aquela que dá origem ao sinal geofísico mais forte ou com maior extensão horizontal, independentemente de ser um alvo geológico de alto valor econômico.
Essa inversão não requer a filtragem prévia do sinal produzido pelas fontes não-alvo para isolar o sinal da fonte alvo.
Isso é realizado por meio da minimização das funções dos vínculos definidas no espaço dos parâmetros e a função de desajuste de dados magnéticos definida no espaço dos dados como a norma-1 dos resíduos.
As soluções estimadas são estabilizadas por funções vínculo definidas pelas regularizações de Tikhonov de ordem zero e de primeira ordem.
A inversão magnética robusta assume que a direção de magnetização total da fonte alvo é conhecida, uniforme e estima: i) a intensidade de magnetização total, ii) as profundidades do topo e da base, iii) a posição e iv) a forma do corpo 3D em profundidade.
A solução regularizada obtida minimizando-se a norma-1 dos resíduos é chamada solução L1.
Para fins de comparação, o método também estima a solução L2 que minimiza as mesmas funções dos vínculos e a função de desajuste dos dados magnéticos definida pela norma-$2$ dos resíduos.

Testes em dados sintéticos mostram que, na presença de fontes não-alvo que produzem anomalias interferentes, a solução L1 mostra um melhor desempenho que a L2 não apenas na recuperação da geometria da fonte alvo, mas também na estimativa dos valores de profundidade do topo e de intensidade de magnetização.
Na ausência de dados magnéticos interferentes, as soluções L1 e L2 apresentam comportamentos semelhantes.

Aplicações a dados aeromagnéticos sobre os complexos alcalinos de Anitápolis, SC, e Diorama, GO, permitiram inferir detalhes sobre seus controles estruturais.
A semelhança entre as soluções L1 e L2 obtidas no complexo Anitápolis sugerem a presença de fontes não-alvo pequenas e rasas.
Ambas as soluções L1 e L2 mostram uma fonte alvo mergulhando para noroeste, com strike definido por um lineamento tectônico conhecido na área.
No caso do complexo alcalino de Diorama, em GO, a solução L1 mostra uma intrusão principal (fonte alvo) sub-vertical, que parece ser controlada por uma falha conhecida com tendência a nordeste.
A solução L2 obtida neste caso se mostrou muito diferente da L1,
o que sugere a presença
de ruído geológico
associado à presença
de fontes não-alvo
relativamente grandes.
Por conta disso, a solução L1 é considerada
a mais plausível.

Sem o conhecimento da direção uniforme de magnetização total da fonte alvo, o método proposto neste trabalho, não é capaz de gerar soluções coerentes com a geologia.
Outra limitação deste método é a definição da aproximação inicial, uma vez que ela depende do cálculo prévio da anomalia RTP, de um ajuste preliminar dos dados observados, além da escolha do número de prismas e vértices.
Considero que esses são pontos a serem explorados futuramente para o aprimoramento do método.
Além disso, um novo desafio seria generalizar o método para estimar múltiplas fontes simultaneamente.
Essa generalização pode ser acompanhada pela inversão conjunta de diversos dados, tais como dados magnéticos, de gravidade e de gradiometria da gravidade.