\chapter{Conclusões}


Do ponto de vista geofísico, os sinais produzidos por fontes não-alvo podem ser considerados um ruído geológico e devem ser filtrados dos sinais produzidos pelas fontes alvo que são de grande interesse em uma interpretação geológica.
Aqui, uma fonte alvo é aquela que dá origem ao sinal geofísico mais forte, independentemente de ser um alvo geológico de alto valor econômico.
Este trabalho apresentou uma inversão regularizada não linear com um ajuste robusto para inverter dados magnéticos a fim de estimar a forma 3D de uma fonte alvo na presença ou não de fontes não-alvo.
Essa inversão não requer filtrar o sinal produzido pelas fontes não-alvo para isolar o sinal das fontes alvo.
Isso é realizado por meio da minimização das funções dos vínculos definidas no espaço dos parâmetros e a função de desajuste de dados magnéticos definida no espaço dos dados como a norma-1 do vetor dos resíduos (a diferença entre os dados observados e preditos).
As soluções estimadas são estabilizadas por funções de vínculos definidas pelas regularização de Tikhonov de ordem zero e de primeira ordem.
A inversão magnética robusta assume a direção de magnetização total da fonte alvo e recupera: i) a intensidade de magnetização total, ii) as profundidades do topo e da base, e iii) as bordas das fatias de profundidade horizontal do corpo 3D.
A solução recuperada é chamada de solução L1 e nós a comparamos com a solução L2 que minimiza as mesmas funções dos vínculos e a função de desajuste dos dados magnéticos definida pela norma-$2$ do vetor dos resíduos.

Testes em dados sintéticos mostram que a fonte estimada (solução L1) recupera a forma 3D de uma fonte alvo, mesmo se haja ou não fontes não-alvo que produzem parte dos dados.
Na presença de dados magnéticos interferentes e em comparação com a solução L2,
a solução L1 mostra um melhor desempenho não apenas na recuperação da geometria da fonte alvo, mas também na estimativa dos valores de profundidade do topo e de intensidade de magnetização total mais próximos dos verdadeiros.
Na ausência de dados magnéticos interferentes, as soluções L1 e L2 apresentam comportamentos semelhantes.
Além disso, quando há ou não a influência de um campo regional intenso sobre os dados, ambas as soluções L1 e L2 foram bem sucedidas em estimar a geometria da fonte alvo que foi simulada como um corpo inclinado.

Aplicações a dados aeromagnéticos de levantamentos sobre complexos alcalinos brasileiros permitiram inferir algumas características geológicas.
A partir dos resultados, é possível afirmar que o complexo alcalino-carbonatítico Anitápolis, em SC, é composto de fontes não alvo e não-alvo, que produzem anomalias magnéticas de fraca interferência.
As soluções L1 e L2 produzem estimativas semelhantes no complexo Anitápolis, as quais sugerem a presença de fontes não-alvo pequenas e rasas.
Em relação à fonte alvo do complexo Anitápolis, ambas as soluções L1 e L2 mostram uma intrusão mergulhando para noroeste que concorda com um lineamento tectônico conhecido na área.
No caso do complexo alcalino de Diorama, em GO, é possível afirmar ele é composto de fontes alvo e não-alvo que produzem anomalias magnéticas de interferência intermediária.
No entanto, as fontes não-alvo do complexo de Diorama não devem ser pequenas porque as soluções L1 e L2 não produzem estimativas semelhantes.
A fonte-alvo do complexo Diorama mostra uma forma com variação de profundidade que parece ser controlada por uma falha conhecida com tendência a nordeste.

Numa perspectiva futura, um novo desafio seria generalizar a inversão apresentada para estimar múltiplas fontes simultaneamente.
Além disso, essa generalização pode ser acompanhada pela inversão conjunta de diversos dados, como dados magnéticos, dados de gravidade e dados de gradiometria gravimétrica.