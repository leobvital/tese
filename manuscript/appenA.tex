\chapter{Algumas Demonstrações}

\section{First section}

Aqui devem entrar demonstrações mais longas, revisões de conceitos mais básicos
ou qualquer detalhe pertinente que não seja adequado para o corpo da
dissertação/tese.

\begin{equation}
\Gamma(x, y, z) = \int\limits_{-\infty}^{+\infty}\int\limits_{-\infty}^{+\infty}
p(x', y', z_{c}) \, \frac{\partial}{\partial z} \frac{1}{r} \: dx' \, dy'
\label{eq:apendice1}
\end{equation}

\section{Second section}

\lipsum[3-5]

\section{Third section}

\begin{table}[h]
	\caption[Cópia da tabela do capítulo 2]
	{Cópia da tabela do capítulo 2 para ilustrar numeração no apêndice}
	\label{tab:citation-copy}
	\centering
	{\footnotesize
		\begin{tabular}{|c|c|c|}
			\hline
			Tipo da Publica{\c c}\~ao & \verb|\citep| & \verb|\citet|\\
			\hline
			Livro & \citep{book-example} & \citet{book-example}\\
			Artigo & \citep{article-example} & \citet{article-example}\\
			Relat\'orio & \citep{techreport-example} & \citet{techreport-example}\\
			Relat\'orio & \citep{techreport-exampleIn} & \citet{techreport-exampleIn}\\
			Anais de Congresso & \citep{inproceedings-example} &
			\citet{inproceedings-example}\\
			S\'eries & \citep{incollection-example} & \citet{incollection-example}\\
			Em Livro & \citep{inbook-example} & \citet{inbook-example}\\
			Disserta{\c c}\~ao de mestrado & \citep{mastersthesis-example} &
			\citet{mastersthesis-example}\\
			Tese de doutorado & \citep{phdthesis-example} & \citet{phdthesis-example}\\
			\hline
	\end{tabular}}
\end{table}